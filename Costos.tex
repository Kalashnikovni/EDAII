\nonstopmode

\usepackage{fullpage}

\documentclass[a4paper,10pt]{article}

\usepackage[utf8]{inputenc}
\usepackage[english]{babel}
\usepackage{amsmath}
\usepackage[margin=1in]{geometry}

\author{
    Marzorati Denise \\
    \texttt{marzorati.denise@gmail.com}
    \and Soncini Nicolás \\
    \texttt{soncininicolas@gmail.com}
}

\date{
    8 de Junio de 2016
}

\title{
    \Huge \textsc{Especificación de costos} \\
    \large \textsc{Estructuras de Datos y Algoritmos II} \\
    \textsc{Trabajo Práctico 2}
}

\begin{document}

\maketitle

\thispagestyle{empty}

\begin{center}
\large \bf Docentes de la materia
\end{center}

\begin{center}
Mauro Jaskelioff

Cecila Manzino

Juan M. Rabasedas

Martin Ceresa
\end{center}

\newpage{}


\part*{Implementación con listas}


%FILTERS
\section*{\texttt{filterS}}

%TODO: Poner sangría acá%
    La implementación de $filterS$ con listas hace uso del paralelismo para
reducir su profundidad lo mas posible en base a las aplicaciones de la función 
que se le pasa.

    El \texttt{trabajo} de la función se puede tomar de la siguiente forma:

\begin{equation*}
    W \left(filterS\; f \;xs\right) \in
    O \left( \sum_{i=0}^{\vert xs \vert -1} W \left( f\; xs_i \right) \right)
\end{equation*}

De esta forma, se calcula la aplicacion de la función argumento en cada
llamada recursiva de $filterS$ hasta que la lista termine vacía.
En el mejor de los casos, si $W \left( f\; xs_i \right) \in O \left( 1 \right)$ queda
su sumatoria como $\vert xs \vert$, por lo tanto se puede omitir sumar este valor
en la cota superior del trabajo.
%TODO: Cambiar esto para que deje espacio y ponga sangría
\bigskip
\\

La \texttt{profundidad} de la función se puede reducir gracias a la aplicación
en paralelo de la función argumento, por lo tanto se puede deducir que:

\begin{equation*}
    S \left(filterS\; f \; xs \right) \in
    O \left( \vert xs \vert + \max_{i=0}^{\vert xs \vert -1} S \left( f\; xs_i \right) \right)
\end{equation*}

En este caso, como el máximo de todos los trabajos puede quedar constante, debemos
sumar el trabajo de obtener cada elemento, lo cual suma $\vert xs \vert$.


\bigskip

 
%SHOWT
\section*{\texttt{showtS}}

    La implementación de $showtS$ con listas divide la lista en dos mitades
y con las funciones $takeS$ y $dropS$

    Sabemos que las funciones utilizadas son lineales:

\begin{equation*}
    W/S \left( takeS\; xs\; n \right) \in 
    O \left( \vert xs \vert \right)
\end{equation*}
\begin{equation*}
    W/S \left( dropS\; xs\; n \right) \in 
    O \left( \vert xs \vert \right)
\end{equation*}
tanto en su trabajo como en su profundidad. 

\bigskip
\\   
    Luego, podemos concluir fácilmente que valen los siguientes costos ya
que se aplica una vez cada función sobre la mitad de la lista.

\begin{equation*}
    W \left( showtS\; xs \right) \in O \left( \vert xs \vert \right)
\end{equation*}

\begin{equation*}
    S \left( showtS\; xs \right) \in O \left( \vert xs \vert \right)
\end{equation*}


\bigskip
    

%REDUCES
\section*{\texttt{reduceS}}

%TODO: En forma de arbol logaritmico?!?!?!
    La implementación de $reduceS$ toma una función de costo desconocido,
un elemento y una lista del mismo tipo y aplica el algoritmo de reduce
de forma que opera con la función dada $\oplus$ los elementos de la lista
en forma de árbol logarítmico (gracias al algoritmo contraer).

    Su \texttt{trabajo} se puede calcular teniendo en cuenta que la implementación
del algoritmo emplea una reducción del tamaño del arreglo en las llamadas a
la función $contraer$, que cumple:

\begin{equation*}
    W \left( contraer\; \oplus \;xs \right) \in
    O \left( \sum_{i=0}^{(\vert xs \vert /2 )-1} xs_{2*i} \oplus xs_{(2*i) +1} \right)
\end{equation*}

\begin{equation*}
    S \left( contraer\; \oplus \;xs \right) \in
    O \left( \right)
\end{equation*}



\begin{equation*}
    W \left( reduceS\; \oplus \;b \;xs \right) \in
    O \left( \right)
\end{equation*}

\begin{equation*}
    S \left( reduceS\; \oplus \;b \;xs \right) \in
    O \left( \right)
\end{equation*}



%SCANS
\section*{\texttt{scanS}}

    La implementación de \texttt{scanS}

\begin{equation*}
    W \left( scanS\; \oplus \;b \;s \right) \in
    O \left( \right) 
\end{equation*}

\begin{equation*}
    S \left( scanS\; \oplus \:b \:s \right) \in
    O \left( \right)
\end{equation*}
% \; is a thick space
% \vert is |
% \left( is big left parenthesis
% \right) is big right parenthesis
% \sum is summatory.
% \max is max (math mode)
% \mathbf is bold formatting

\begin{equation*}
    W \left( filterS\; f \; s \right) \in
    O \left( \sum_{i=0}^{\vert s \vert -1} W \left( f \; s_i \right) \right)
\end{equation*}

\begin{equation*}
    S \left( filterS\; f \; s \right) \in
    O \left( \vert s \vert + \max_{i=0}^{\vert s \vert -1} S \left( f \; s_i \right) \right)
\end{equation*}


\section*{\texttt{showtS}}

\texttt{showtS} precisa partir la secuencia dada en dos mitades, de modo que
utiliza las funciones \texttt{takeS} y \texttt{dropS} (que en el caso de la
implementación con listas son las mismas funciones que hay en \texttt{Prelude})
que tienen orden lineal. Esto resulta en costo lineal para trabajo y profundidad:

\begin{equation*}
    W \left( showtS \; s \right) \in
    O \left( \vert s \vert \right)
\end{equation*}

\begin{equation*}
    S \left( showtS \; s \right) \in
    O \left( \vert s \vert \right)
\end{equation*}


\section*{\texttt{reduceS}}

Para la implementación de \texttt{reduceS} con listas se usa la función \texttt{contract}
que toma una operación binaria $\oplus$ y una secuencia $s$ y la contrae aplicando
la operación binaria a cada par de elementos contiguos. Para esto se calcula la
operación entre dichos elementos en paralelo al cálculo del paso recursivo de
\texttt{contract}, resultando en la suma de los trabajos de $\oplus$ aplicado a
cada par de elementos contiguos de la secuencia para el trabajo y la longitud de
la secuencia más la máxima de las profundidades de $\oplus$ aplicado a cada par de
elementos contiguos de la secuencia para la profundidad:

\begin{equation*}
    W \left( contract \oplus s \right) \in
    O \left( \vert s \vert + \sum_{i=0}^{\frac{\vert s \vert}{2} + 1} W \left( s_{2i} \oplus s_{2i+1} \right) \right)
\end{equation*}

\begin{equation*}
    S \left( contract \oplus s \right) \in
    O \left( \vert s \vert + \max_{i=0}^{\frac{\vert s \vert}{2} + 1} S \left( s_{2i} \oplus s_{2i+1} \right) \right)
\end{equation*}

Luego, \texttt{reduceS} aplicado a una operación binaria $\oplus$, un elemento $b$,
y una secuencia $s$, se calcula aplicando recursivamente \texttt{reduceS} a la
misma operación $\oplus$, el mismo elemento $b$ y la secuencia obtenida de aplicar
\texttt{contract} a $s$, generando el orden de reducción buscado. Esto es lo mismo
para el trabajo y la profundidad, de modo que el costo de recorrer todo el árbol
de reducción es proporcional al tamaño de la secuencia (lineal) pues estamos
trabajando con listas. Luego, para el trabajo queda la suma de los trabajos de
cada aplicación de $\oplus$ en el árbol de reducción más el tamaño de la secuencia y
para la profundidad resulta la suma de las profundidades de cada aplicación de
$\oplus$ en el árbol de reducción más el tamaño de la secuencia, ya que por cómo hay
que forzar el orden de reducción no se puede aprovechar la profundidad de \texttt{contract}.

\newpage

\begin{equation*}
    W \left( reduceS \oplus \; b \; s \right) \in
    O \left( \vert s \vert + \sum_{(x \oplus y) \in \mathcal{O}_r(\oplus,b,s)} W \left( x \oplus y \right) \right)
\end{equation*}

\begin{equation*}
    S \left( reduceS \oplus \; b \; s \right) \in
    O \left( \vert s \vert + \sum_{(x \oplus y) \in \mathcal{O}_r(\oplus,b,s)} S \left( x \oplus y \right) \right)
\end{equation*}

\section*{\texttt{scanS}}

\texttt{scanS}, además de utilizar \texttt{contract}, emplea la función \texttt{combine},
que toma una operación binaria $\oplus$, 2 secuencias, $s$ y $s'$, y una bandera
que indica si el índice actual de la secuencia resultante es par (para construir
el orden de reducción buscado). Cuando el índice es par (bandera en $True$),
devuelve $s'_{\frac{i}{2}}$, mientras que cuando el índice es impar devuelve
$s'_{\lfloor \frac{i}{2} \rfloor} \oplus s_{i-1}$ y en este caso ésta operación
se efectúa en paralelo al paso recursivo, de modo que se mejora un poco la
profundidad, quedando:

\begin{equation*}
    W \left( combine \oplus s \; s' \right) \in
    O \left( \vert s \vert + \sum_{i=1}^{\frac{\vert s \vert}{2}} W \left( s'_{i} \oplus s_{2i - 1} \right) \right)
\end{equation*}

\begin{equation*}
    S \left( combine \oplus s \; s' \right) \in
    O \left( \vert s \vert + \max_{i=1}^{\frac{\vert s \vert}{2}} S \left( s'_{i} \oplus s_{2i - 1} \right) \right)
\end{equation*}

Luego, \texttt{scanS} simplemente aplica \texttt{combine} con la operación binaria
$\oplus$, la secuencia original, y la secuencia obtenida de aplicar \texttt{scanS}
a la contracción de la secuencia original con la operación anterior. Dado que
\texttt{contract} tiene trabajo por lo menos lineal y \texttt{combine} también,
el trabajo de scan resulta lineal en la longitud de la secuencia más el trabajo
de todas las aplicaciones de la operación binaria en el árbol de reducción. En
cuanto a la profundidad, \texttt{contract} y \texttt{combine} también se comportan
linealmente por lo menos, de modo que no se puede aprovechar lo poco que se ganó
en profundidad, así que resulta la longitud de la secuencia más la suma de todas
las profundidades de las aplicaciones de la operación binaria en el árbol de
reducción de \texttt{scanS}:

\begin{equation*}
    W \left( scanS \oplus b \; s \right) \in
    O \left( \vert s \vert + \sum_{(x \oplus y) \in \mathcal{O}_s(\oplus,b,s)} W \left( x \oplus y \right) \right)
\end{equation*}

\begin{equation*}
    S \left( scanS \oplus b \; s \right) \in
    O \left( \vert s \vert + \sum_{(x \oplus y) \in \mathcal{O}_s(\oplus,b,s)} S \left( x \oplus y \right) \right)
\end{equation*}

\newpage{}
































\part*{Implementación con arreglos persistentes}

\section*{\texttt{filterS}}

Para la implementación con arreglos persistentes de \texttt{filterS} primero se
transforma la secuencia dada en una secuencia donde cada elemento es un singleton
cuando el elemento cumple con el predicado, o una secuencia vacía cuando no cumple
con el predicado. Esto se logra usando \texttt{tabulate} con una función que tiene
costo igual al del predicado, para luego aplicar flatten a la lista resultante,
de tamaño $\vert s \vert$, donde cada elemento es una secuencia de longitud 1 o 0.
Luego, \texttt{filterS} resulta con trabajo igual a la suma de los trabajos de
cada aplicación del predicado a los elementos de la secuencia, y profundidad igual
al logaritmo de la longitud de la secuencia más el máximo costo de las profundidades
de todas las aplicaciones del predicado a los elementos de la secuencia:

\begin{equation*}
    W \left( filterS \; f \; s \right) \in
    O \left( \sum_{i=0}^{\vert s \vert -1} W(f \; s_i) \right)
\end{equation*}

\begin{equation*}
    S \left( filterS \; f \; s \right) \in
    O \left( \text{lg} \; \vert s \vert + \max_{i=0}^{\vert s \vert -1} S(f \; s_i) \right)
\end{equation*}


\section*{\texttt{showtS}}

La implementación de \texttt{showtS} con arreglos persistentes usa las funciones
\texttt{takeS} y \texttt{dropS} (ambas con trabajo y profundidad $O \left( 1 \right)$,
ya que usan \texttt{subArray}), de modo que el trabajo y la profundidad resulta
constante:

\begin{equation*}
    W \left( showtS \; s \right) \in
    O \left( 1 \right)
\end{equation*}

\begin{equation*}
    S \left( showtS \; s \right) \in
    O \left( 1 \right)
\end{equation*}


\section*{\texttt{reduceS}}

Para la implementación de \texttt{reduceS} con arreglos persistentes definimos
una función \texttt{contract} que contrae la secuencia dada con una operación
binaria utilizando la función \texttt{tabulate} provista, que a su vez utiliza
una función con el mismo costo que la operación binaria. En el caso del trabajo,
\texttt{contract} tiene costo igual a la suma de los trabajos de las aplicaciones
de la operación binaria a cada par de elementos consecutivos. Por otro lado,
la profundidad simplemente tiene costo igual a la máxima profundidad de las
aplicaciones de la operación binaria a cada par de elementos consecutivos:

\begin{equation*}
    W \left( contract \oplus s \right) \in
    O \left( \sum_{i=0}^{\frac{\vert s \vert}{2} + 1} W \left( s_{2i} \oplus s_{2i+1} \right) \right)
\end{equation*}

\begin{equation*}
    S \left( contract \oplus s \right) \in
    O \left( \max_{i=0}^{\frac{\vert s \vert}{2} + 1} S \left( s_{2i} \oplus s_{2i+1} \right) \right)
\end{equation*}

Luego, \texttt{reduceS} utiliza \texttt{contract} en cada llamado recursivo,
transformando la secuencia dada en otra con la mitad de longitud original, hasta
llegar a un singleton. En el caso del trabajo, esto resulta en recorrer la secuencia
en cada paso recursivo (que esto resulta lineal, basta ver que es una recurrencia
dada por $T \left( n \right) = T \left( \frac{n}{2} \right) + n$ donde $n$ es el
tamaño de la secuencia) más la suma de todos los trabajos de la aplicación de la
operación binaria en el árbol de reducción. Por otro lado, la profundidad resulta
en lg $\vert s \vert$ (debido a que cada aplicación de \texttt{contract} resulta
en una secuencia con tamaño igual a la mitad de la original) multiplicado por la
máxima profundidad de todas las aplicaciones de la operación binaria en el árbol
de reducción (que esto basta para especificar una cota superior, si quisiéramos
acotarlo más obtendríamos la suma de las profundidades de todas las aplicaciones
de la operación binaria en cada nivel del árbol de reducción, lo que resulta
bastante engorroso de escribir, por eso se elije la profundidad dada que aún así
es buena cota).

\begin{equation*}
    W \left( reduceS \oplus b \; s \right) \in
    O \left( \vert s \vert + \sum_{(x \oplus y) \in \mathcal{O}_r(\oplus,b,s)} W \left( x \oplus y \right) \right)
\end{equation*}

\begin{equation*}
    S \left( reduceS \oplus b \; s \right) \in
    O \left( \text{lg} \; \vert s \vert \; \max_{(x \oplus y) \in \mathcal{O}_r(\oplus,b,s)} S \left( x \oplus y \right) \right)
\end{equation*}

\section*{\texttt{scanS}}

La implementación con arreglos persistentes de \texttt{scanS}, además de utilizar
\texttt{reduceS}, emplea la función \texttt{combine} que es la que se encarga de
construír el orden de reducción buscado.
\texttt{combine} toma una operación binaria, $\oplus$ y 2 secuencias, $s$ y $s'$,
donde $s'$ se espera que sea el resultado de aplicar \texttt{scanS} con $\oplus$,
un elemento, y la contracción de $s$ con la operación binaria anterior. Ésta función
utiliza \texttt{tabulate} con la operación $\oplus$ y una función auxiliar que
para los índices pares devuelve $s'_{\frac{i}{2}}$ y para los índices impares
devuelve $s'_{\lfloor \frac{i}{2} \rfloor} \oplus s_{i-1}$, que claramente
aporta al trabajo y a la profundidad sólo en éste último caso, de modo la especificación
de costo para \texttt{combine} queda así: 

%La implementación con arreglos persistentes de \texttt{scanS} también utiliza
%la función \texttt{contract} definida por nosotros (cuyo costo está especificado
%en el análisis de \texttt{reduceS}) y una nueva función \texttt{combine} que lo
%que hace es fusionar una secuencia $s$ y otra secuencia $s'$ mediante la operación
%binaria provista, sabiendo que $s'$ es el resultado de \texttt{scanS} aplicado a
%$contract s$
%\texttt{combine} utiliza \texttt{tabulate} con una función que tiene costo $O \left( 1 \right)$
%cuando el índice es par y $O \left( s'_{i} \oplus s_{2i-1} \right)$ cuando el índice 
%es impar, de manera que el trabajo y profundidad dependen de éste último caso,
%quedando:

\begin{equation*}
    W \left( combine \oplus s \; s' \right) \in
    O \left( \vert s \vert + \sum_{i=1}^{\frac{\vert s \vert}{2}} W \left( s'_{i} \oplus s_{2i-1} \right) \right)
\end{equation*}

\begin{equation*}
    S \left( combine \oplus s \; s' \right) \in
    O \left( \max_{i=1}^{\frac{\vert s \vert}{2}} S \left( s'_{i} \oplus s_{2i-1} \right) \right)
\end{equation*}

Por último, \texttt{scanS} simplemente aplica \texttt{combine} a la secuencia original
y a la obtenida al aplicar \texttt{scanS} con la misma operación, el mismo elemento,
y la secuencia original contraída. De esta forma, en el caso del trabajo tiene
que recorrer toda la secuencia para poder calcular la contracción, además de la
suma de todos los trabajos aportados por las aplicaciones de la operación binaria
en la reducción de \texttt{scanS}. Por otro lado, en el caso de la profundidad
sucede algo similar a lo que pasó con \texttt{reduceS} (ver apartado anterior) de
modo que resulta lg $\vert s \vert$ multiplicado por la máxima profundidad de
todas las aplicaciones de la operación binaria en la reducción de \texttt{scanS},
quedando:

\begin{equation*}
    W \left( scanS \oplus b \; s \right) \in
    O \left( \vert s \vert + \sum_{(x \oplus y) \in \mathcal{O}_s(\oplus,b,s)} W \left( x \oplus y \right) \right)
\end{equation*}

\begin{equation*}
    S \left( scanS \oplus b \; s \right) \in
    O \left( \text{lg} \; \vert s \vert \; \max_{(x \oplus y) \in \mathcal{O}_s(\oplus,b,s)} S \left( x \oplus y \right) \right)
\end{equation*}

\end{document}
